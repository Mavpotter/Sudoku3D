\documentclass[a4paper,12pt]{scrreprt}

\usepackage{german}

\title{Sudoku3D}
\subtitle{Projekt 3. Lehrjahr MTS51}
\author{Niclas Hofmann, Son Le Cong \& Marvin Mahn}
\date{29. Mai 2018}


\begin{document}
	\maketitle
	\tableofcontents
	
	\chapter{Einf\"uhrung}
	\section{Vorwort}
	Die vorliegende Dokumentation wurde von den Autoren im Rahmen des Projektes im dritten Lehrjahres
	f\"ur das Lernfeld 13 erstellt.\medskip \\
	Zur Unterscheidung, welche Texte von welchen Autoren verfa{\ss}t wurden, wird dies an den jeweiligen
	Stellen erw\"ahnt\footnote{Ist kein Name angef\"uhrt war der Autor Niclas Hofmann}.
	Sinnvollerweise beschreibt jeder Autor die von ihm erstellten Teile des Projektes.
	
	\section{Problemdefinition}
	Herk\"ommliche Sudokus k\"onnen je nach Schwierigkeitsgrad zwar an Komplexit\"at gewinnen,
	bleiben aber dennoch recht schnell l\"osbar. Auch Sonderformen wie Killer-Sudokus\footnote{
	https://de.wikipedia.org/wiki/Killer-Sudoku} oder X-Sudokus\footnote{https://de.wikipedia.org/
	wiki/Sudoku\#X-Sudoku} bieten nur bis zu einem gewissen Grad weiteren Anreiz.

	\section{Projektziel}
	Um etwas neuartiges zu schaffen, das genau diesen Anreiz bietet, soll nun ein Programm erstellt
	werden, welches 3D-Sudokus kreiert, analog zum 3D-Schach aus der Fernsehserie Star Trek.
	\medskip \\
	Aber auch hier soll es m\"oglich sein, nicht nur mit den Standardregeln zu spielen, sondern auch
	andere verwenden zu k\"onnen.\medskip \\
	Au{\ss}erdem soll das Endprodukt als interaktives Spiel auf der MATSE-Hompage\footnote{
	https://www.matse-ausbildung.de/} angeboten werden.

	\section{Wichtige Termine}
	Diese Projekt mu{\ss} am 29. Mai 2018 pr\"asentiert werden.
	
	\chapter{Entwicklerdokumentation}
	\section{Model}
	\subsection{Problemanalyse}
	F\"ur das Model fallen die folgenden Aufgaben an:
	\begin{enumerate}
		\item Erstellung eines Sudoku3D's nach den ausgew\"ahlten Regelmodulen
		\item Entfernung einiger Zahlen, um ein spielbares Sudoku3D anzubieten
		\item Speicherung der Daten (vorgegebene Zahlen und User-Eingaben)
		\item \"Uberpr\"ufung, ob das Sudoku3D gel\"ost wurde
		\item Ein $clear$, das alle User-Eingaben entfernt
		\item Die Zur\"uckgabe aller falsch eingetragenen zahlen
		\item Wiederherstellung eines Sudoku3D mit gegebenen Zahlen, Regelmodulen und
			User-Eingaben
	\end{enumerate}

	\subsubsection{Regelmodule}
	Da zum Erstellen eines Sudoku3D's Regelmodule ben\"otigt werden, m\"u{\ss}en diese vorhanden
	sein. Doch was ist/leistet ein Regelmodul?\medskip \\
	Ein Regelmodul definiert alle Einheiten (Units), in denen die Zahlen von 1-9 nur einmal vorkommen
	d\"urfen. Bei einem normalen Sudoku sind dies z. B. die 3x3 Boxen, die Zeilen und die Spalten.
	Weil die Unterscheidung zwischen Zeile und Spalte in der Dimension liegt, werden diese im
	Sudoku3D nicht unterschieden. Somit sind die zwei Grundlegenden Regelmodule auch schon vorgegeben.s
\end{document}